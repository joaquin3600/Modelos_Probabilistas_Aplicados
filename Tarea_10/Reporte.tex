
\documentclass[]{article}

\title{Practica Ejercicios en R}

\date{}
\usepackage{braket}
\usepackage{bbold}
\usepackage{amsmath,amsfonts,amssymb,amsthm,booktabs}
\usepackage[margin=1.0in]{geometry}
\usepackage{graphicx}
\usepackage{chngcntr}
\usepackage{floatrow}
\usepackage{chngcntr}
\usepackage{hyperref}
\usepackage[spanish]{babel}
\usepackage[svgnames]{xcolor}

\usepackage{floatrow}
\floatsetup[table]{capposition=top}

\renewcommand{\spanishtablename}{Cuadro}
\usepackage{listings}
\usepackage[%
    font={small,sf},
    labelfont=bf,
    format=hang,    
    format=plain,
    margin=0pt,
    width=0.8\textwidth,
]{caption}
\usepackage[list=true]{subcaption}
\lstset{language=R,
    basicstyle=\small\ttfamily,
    stringstyle=\color{DarkGreen},
    otherkeywords={0,1,2,3,4,5,6,7,8,9},
    morekeywords={TRUE,FALSE},
    deletekeywords={data,frame,length,as,character},
    keywordstyle=\color{blue},
    commentstyle=\color{DarkGreen},
}

\counterwithin{figure}{section}
\renewcommand*{\figureautorefname}{Figura}


\usepackage[backend=biber]{biblatex}
\addbibresource{ref.bib}

\begin{document}
	\maketitle
	\begin{center}


\centerline{\textbf{TAREA 10} } 
\textbf{ }

\centerline{Alumno: } 
\centerline{Joaquín Arturo Velarde Moreno}


	\end{center}
	

\section{Introducción}
En este reporte hago uso del programa R 4.0.2 \cite{rproject} para poder resolver 3 problemas escogidos del libro de introducción a la probabilidad \cite{Material} y representare gráficamente los resultados para ver como se comportan las variables.


\section{Ejercicio 1, página 247}


Se saca una carta al azar de una baraja que consta de cartas numeradas del 2 al 10. Un jugador gana 1 dólar si el número de la carta es impar y pierde 1 dólar si el número es par. ¿Cuál es el valor esperado de sus ganancias?\\
$E = \sum x *P(X=x)$ .\\
$E = (-1)P(x=-1) + 1(1)P(x=1)$.\\
$P(X=1) = \frac{4}{9}$ .\\
$P(X=-1) = \frac{4}{9}$ .\\
$E = (-1)(\frac{4}{9}) +(1)(\frac{4}{9})$. \\
$E = -\frac{1}{9}$ .\\
Analíticamente obtenemos que nuestro valor esperado es de $-\frac{1}{9}$ o $-0.11111111$.
Ahora pasaremos a aplicarlo en nuestro programa R
,de acuerdo a nuestro problema tenemos una baraja con cartas numeradas del 2 al 10.
   \begin{lstlisting}
	Baraja <- c(2,3,4,5,6,7,8,9,10)
   \end{lstlisting}
de acuerdo a la instrucción debemos sacar una carta al azar de la baraja ,para esto haremos uso de la función \textit{sample()}.
   \begin{lstlisting}
	carta  <- sample(Baraja, 1)
   \end{lstlisting}
Lo siguiente es validar si nuestro numero es par o impar para esto usaremos la operación módulo y lo guardaremos en un arreglo.
   \begin{lstlisting}
      if((carta %% 2) == 0)
      {
        Ganancias = - 1;
      }else {
        Ganancias =   1;
      }
	  ArregloGanancias = c(ArregloGanancias,Ganancias)

   \end{lstlisting}
   Por ultimo replicaremos nuestro proceso cuantas veces queramos para comprobar si nuestro valor esperado es obtenido.
      \begin{lstlisting}
  for(j in 1:Replicas)
  {
    Ganancias <- 0;
    for(i in 1:50)
    {
      carta  <- sample(Baraja, 1)
      if((carta %% 2) == 0)
      {
        Ganancias = - 1;
      }else {
        Ganancias =   1;
      }
      ArregloGanancias = c(ArregloGanancias,Ganancias)
    }
    Media = c(Media,mean(ArregloGanancias))
  }

      \end{lstlisting}
      Con esto nos dará resultado que ha medida que el numero de replicas aumenta el valor de nuestra media se aproxima mas a $-\frac{1}{9}$ (\autoref{fig:casos}).
\begin{figure}[hbt!]
\centering
\subcaptionbox{Media del evento replicada 10 veces.}{\includegraphics[width=0.3\textwidth]{Figuras/boxplot10.png}}%
\hfill
\subcaptionbox{Media del evento replicada 100 veces.}{\includegraphics[width=0.3\textwidth]{Figuras/boxplot100.png}}%
\hfill
\subcaptionbox{Media del evento replicada 1000 veces.}{\includegraphics[width=0.3\textwidth]{Figuras/boxplot1000.png}}%
\hfill
\subcaptionbox{Media del evento replicada 10 veces.}{\includegraphics[width=0.3\textwidth]{Figuras/hist10.png}}%
\hfill
\subcaptionbox{Media del evento replicada 100 veces.}{\includegraphics[width=0.3\textwidth]{Figuras/hist100.png}}%
\hfill
\subcaptionbox{Media del evento replicada 1000 veces.}{\includegraphics[width=0.3\textwidth]{Figuras/hist1000.png}}%
\hfill
\caption{Distribuciones de medias en los experimentos.}

\label{fig:casos}
\end{figure}




    
\section{Ejercicio 18, página 249}


Exactamente una de las seis llaves similares abre una puerta determinada. Si prueba las teclas, una tras otra, ¿cuál es la cantidad esperada de teclas que tendrá que probar antes de tener éxito?\\
$E(X) = \sum_{x = 1}^{6} x * P(X = x)$.\\
$E(X) = \sum_{x = 1}^{6} x * \frac{1}{6}$.\\
$E(X) = (1+2+3+4+5+6)* \frac{1}{6}$.\\
$E(X) = 3.5$.\\
Analíticamente obtenemos que nuestro valor esperado es de $3.5$.
Para este ejercicio tomaremos un vector de 5 elementos para representar nuestras 6 llaves y estableceremos una llave correcta.
      \begin{lstlisting}
		Llaves         <- c(1:6)
		LlaveCorrecta  <- sample(Llaves, 1)

      \end{lstlisting}
Después recorreremos nuestro vector en busca de la llave correcta que tengamos, contando la veces que nos tomo encontrarla.
      \begin{lstlisting}
	Llaves         <- c(1:6)
	LlaveCorrecta  <- sample(Llaves, 1)
    for (LLave in LLaves) 
    {
      if(LLave == LLaveCorrecta)
      {
        break;
      }else{
        contador = contador + 1;
      }
    }
      \end{lstlisting}
Por ultimo sacaremos la media de las veces que nos tomo sacarlas para crear así un arreglo de medias.
           \begin{lstlisting}
  for(i in 1:100)
  {
    LLaveCorrecta  <- sample(LLaves, 1)
    contador       <- 0;
    for (LLave in LLaves) 
    {
      if(LLave == LLaveCorrecta)
      {
        break;
      }else{
        contador = contador + 1;
      }
    }
    contadores = c(contadores,contador)
  }
  Medias  <- c(Medias,mean(contadores))
      \end{lstlisting}
      
 Para mi sorpresa el valor de la media obtenido mas frecuentemente es el de $2.5$ por lo que mas pruebas son necesarias.
 \begin{figure}[hbt!]
\centering
\subcaptionbox{Frecuencia de medias .}{\includegraphics[width=0.4\textwidth]{Figuras/histllave.png}}%
\hfill
\subcaptionbox{Distribuciones de la media.}{\includegraphics[width=0.4\textwidth]{Figuras/boxplotllave.png}}%
\hfill


\caption{Distribución de la media para el numero de llaves necesarias en 10 eventos.}

\label{fig:casos2}
\end{figure}
     
\section{Ejercicio 1, página 263}



Se elige un número al azar del conjunto $ S = \{- 1, 0, 1\} $. Sea $ X $ el número elegido. Encuentre el valor esperado, la varianza y la desviación estándar de $ X $.\\
$E(X) = \sum_{x = 1}^{3} x * P(X = x)$.\\
$E(X) = \sum_{x = 1}^{3} x * \frac{1}{3}$.\\
$E(X) = (-1, 0, 1)* \frac{1}{3}$.\\
$E(X) = 0$.\\
ahora para la varianza:
$\sigma^{2} = V(X) = E((x-E(x))^{2})$.\\
$\sigma^{2} = E((x-0)^{2})$.\\
$\sigma^{2} = E(x^{2})$.\\
$\sigma^{2} = \sum x^{2} * P(X^{2} = x^{2})$.\\
$\sigma^{2}  = (0 * \frac{1}{3}) + (1 * \frac{2}{3})$.\\
$\sigma^{2}  = \frac{2}{3}$.\\
por ultimo la desviación estándar:
$\sigma = \sqrt{\frac{2}{3}}$.\\
Viendo que analíticamente conseguimos nuestro valor de la media como $0$, la varianza como $0,6$ y la desviación estándar como $0,81$, aplicaremos el ejercicio en R , primero necesitamos un arreglo que vaya de -1 a 1 y obtener un numero al azar.
      \begin{lstlisting}
    Numeros   <- c(-1,0,1)
    Numero    <- sample(Numeros, 1)

      \end{lstlisting}
Haremos este proceso 100 veces y guardaremos nuestros resultados en un vector.
      \begin{lstlisting}
Numeros         <- c(-1,0,1)

  DistribucionNumeros <- c()
  for(i in 1:100)
  {
    Numero    <- sample(Numeros, 1)
    DistribucionNumeros <- c(DistribucionNumeros,Numero)
  }

      \end{lstlisting}
Por ultimo obtendremos la media, la varianza y la desviación estándar de cada uno y repetiremos este proceso 1000 veces para ver el resultado.

      \begin{lstlisting}
Numeros         <- c(-1,0,1)
for(j in 1:1000)
{
  DistribucionNumeros <- c()
  for(i in 1:100)
  {
    Numero    <- sample(Numeros, 1)
    DistribucionNumeros <- c(DistribucionNumeros,Numero)
  }
  ArrayMedias     = c(ArrayMedias,mean(DistribucionNumeros))
  ArrayVarianza   = c(ArrayVarianza,var(DistribucionNumeros))
  ArrayDestandar  = c(ArrayDestandar,sd(DistribucionNumeros))
}

      \end{lstlisting}
Con esto podemos ver que nuestros valores se acercan a los obtenidos analíticamente \autoref{fig:casos3}.
 \begin{figure}[hbt!]
\centering
\subcaptionbox{Frecuencia de medias obteniendo en promedio $0$.}{\includegraphics[width=0.3\textwidth]{Figuras/histmedia.png}}%
\hfill
\subcaptionbox{Frecuencia de la varianza obteniendo en promedio $0.6$.}{\includegraphics[width=0.3\textwidth]{Figuras/histvarianza.png}}%
\hfill
\subcaptionbox{Frecuencia de la desviación estándar obteniendo en promedio $0.81$.}{\includegraphics[width=0.3\textwidth]{Figuras/histDE.png}}%
\hfill

\caption{Frecuencias de la media, varianza y desviación estándar.}

\label{fig:casos3}
\end{figure} 
\printbibliography[title={Referencias}]
\end{document}
