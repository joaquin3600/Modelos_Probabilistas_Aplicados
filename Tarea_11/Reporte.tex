
\documentclass[]{article}

\title{Tarea convolución}

\date{}
\usepackage{braket}
\usepackage{bbold}
\usepackage{amsmath,amsfonts,amssymb,amsthm,booktabs}
\usepackage[margin=1.0in]{geometry}
\usepackage{graphicx}
\usepackage{chngcntr}
\usepackage{floatrow}
\usepackage{chngcntr}
\usepackage{hyperref}
\usepackage[spanish]{babel}
\usepackage[svgnames]{xcolor}

\usepackage{floatrow}
\floatsetup[table]{capposition=top}

\renewcommand{\spanishtablename}{Cuadro}
\usepackage{listings}
\usepackage[%
    font={small,sf},
    labelfont=bf,
    format=hang,    
    format=plain,
    margin=0pt,
    width=0.8\textwidth,
]{caption}
\usepackage[list=true]{subcaption}
\lstset{language=R,
    basicstyle=\small\ttfamily,
    stringstyle=\color{DarkGreen},
    otherkeywords={0,1,2,3,4,5,6,7,8,9},
    morekeywords={TRUE,FALSE},
    deletekeywords={data,frame,length,as,character},
    keywordstyle=\color{blue},
    commentstyle=\color{DarkGreen},
}

\counterwithin{figure}{section}
\renewcommand*{\figureautorefname}{Figura}

\usepackage[backend=biber]{biblatex}
\addbibresource{ref.bib}

\begin{document}
	\maketitle
	\begin{center}


\centerline{\textbf{TAREA 10} } 
\textbf{ }

\centerline{Alumno: } 
\centerline{Joaquín Arturo Velarde Moreno}


	\end{center}
	

\section{Introducción}
En este reporte hago uso del programa R 4.0.2 \cite{rproject} para poder demostrar y probar algunas propiedades de la covarianza propuestos del material del curso\cite{MaterialClase}.


\section{Covarianza}

La covarianza es un dato básico que existe para determinar una dependencia entre 2 variables aleatorias, a diferencia de los coeficientes de correlación este no esta estandarizado, por lo que puede tomar valores de $\infty$ hasta $-\infty$ y lo representamos como $Cov(X,Y)$, esto es igual a $E[(X - E[X])(Y - E[Y])]$ por lo tanto:
\[Cov(X,Y) = E[(X - E[X])(Y - E[Y])]. \]
\subsection{Primera propiedad}
Probaremos la siguiente propiedad de la covarianza, de manera numérica empleando la herramienta R \cite{rproject}, si $X,Y$ son variables aleatorias y $a, b, c, d$ son constantes tenemos que:
\[Cov(aX + b,cY + d) = acCov(X,Y). \]
Primero estableceremos nuestras variables aleatorias por medio de una distribución uniforme para $X$ y asignando a nuestra $Y$ una operación de $X$.
      \begin{lstlisting}
        	X <- runif(100)
		Y <- X*2/3
      \end{lstlisting}
después declaramos nuestras constantes. 
      \begin{lstlisting}
        	X <- runif(100)
		Y <- X*2/3
		a <- 1
		b <- 2
		c <- 3
		d <- 4
      \end{lstlisting}
Ahora representaremos el primer miembro de nuestra ecuación $Cov(aX + b,cY + d)$.
      \begin{lstlisting}
        	X <- runif(100)
		Y <- X*2/3
		a <- 1
		b <- 2
		c <- 3
		d <- 4
		PrimerMiembro = cov((a * X) + b, (c * Y) + d)					
      \end{lstlisting}
Por ultimo obtenemos nuestro segundo miembro de nuestra ecuación $acCov(X,Y)$.
      \begin{lstlisting}
        	X <- runif(100)
		Y <- X*2/3
		a <- 1
		b <- 2
		c <- 3
		d <- 4
		PrimerMiembro  = cov((a * X) + b, (c * Y) + d)
		SegundoMiembro = a * c * cov(X,Y)
		#> print(PrimerMiembro)
        	#[1] 0.1518915
        	#> print(SegundoMiembro)
        	#[1] 0.1518915
      \end{lstlisting}
con esto podemos demostrar que ambos miembros de la ecuación son lo mismo, para la prueba analítica tenemos que recordar que la $Cov(X,Y) $ es $E(XY) - E(X)E(Y)$. Si tenemos nuestras variables siendo afectadas por las constantes entonces tenemos que.\\


\begin{tabular}{c}

$\begin{array} {lcl} 
Cov(aX + b,cY + d) 
& = & E[(aX + b)(cY + d)] - E(aX + b)E(cY + d) \\ 
& = & E(acXY + adX + bcY + bd) - (acE(X)E(Y)+cbE(Y)+adE(X)+bd)  \\ 
& = & acE(XY)+ adE(Y)+ bcE(Y)+ bd - (acE(X)E(Y)+ cbE(Y) + adE(Y) + bd)  \\ 
& = & acE(XY)- acE(x)E(Y)  \\ 
& = & ac(E(XY) - E(X)E(Y))  \\ 
& = & acCov(X,Y)  \\ 
\end{array}$  \\ 
 
\end{tabular}

por lo que nos queda que nuestra propiedad analíticamente es correcta.

\subsection{Segunda propiedad}
Probaremos ahora una segundo propiedad de la varianza, de manera numérica empleando la herramienta R \cite{rproject}, si $X,Y$ son variables aleatorias, entonces la varianza de la suma de las variables es igual a la varianza de $X$ y $Y$ mas 2 veces la covarianza de $X,Y$  :
\[Var(X + Y) = Var(X) + Var(Y) + 2Cov(X,Y). \]
Primero estableceremos nuestras variables aleatorias por medio de una distribución uniforme para $X$ y asignando a nuestra $Y$ una operación de $X$.
      \begin{lstlisting}
        	X <- runif(100)
		Y <- X*2/3
      \end{lstlisting}

Ahora representaremos los miembros de nuestra ecuación $Var(X) + Var(Y) + 2Cov(X,Y)$.
      \begin{lstlisting}
        	X <- runif(100)
		Y <- X*2/3
		PrimerMiembro   = var(X + Y)
		SegundoMiembro  = var(X) + var(Y) + (2 * cov(X,Y))
		#> print(PrimerMiembro)
        	#[1] 0.2186215
        	#> print(SegundoMiembro)
        	#[1] 0.2186215
      \end{lstlisting}
con esto podemos demostrar que ambos miembros de la ecuación son lo mismo, para la prueba analítica tenemos que recordar que la $V(X) $ es $E((X + Y)^{2}) - (E(X + Y))^{2}$. Si tenemos nuestras sumadas tenemos que:\\


\begin{tabular}{c}

$\begin{array} {lcl} 
V(X + Y) 
& = & E[(X + Y)^{2}] - (E(X + Y))^{2}\\ 
& = & E(X^{2} + 2XY + Y^{2}) - (E(X) + E(Y))^{2}  \\ 
& = & E(X^{2})+2E(XY)+E(Y^{2}) - (E(X))^{2} + (E(Y))^{2} + 2E(X)E(Y)\\ 
& = & E(X^{2})-(E(X))^{2} + E(Y^{2}) - (E(Y))^{2}  \\ 
& = & V(X) + V(Y) + 2Cov(X,Y)  \\ 
\end{array}$  \\ 
 
\end{tabular}      

por lo que nos queda que nuestra propiedad analíticamente es correcta.      
\hfill
\printbibliography[title={Referencias}]
\end{document}
